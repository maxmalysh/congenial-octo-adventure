%%%%%%%%%%%%%%%%%%%%%%%%%%%%%%%%%%%%%
\subsection{Стационарные итерационные процессы}
Были реализованы следующие стационарные итерационные процессы, используемые для решения систем вида $Ax=b$:

\begin{enumerate}
	\item метод Якоби;
	\item метод Гаусса-Зейделя;
	\item метод последовательной релаксации (SOR);
	\item метод симметричной последовательной релаксации (SSOR);
\end{enumerate}

Данные итерационные процессы исследовались на предмет скорости сходимости. Для остановки использовался следующий критерий сходимости:

\[
\frac{||r||}{||b||} \leq tol_{rel}
\]
где $tol_{rel}$ – заданный параметр допустимой относительной погрешности, $r$ - вектор невязки $r = b - A\hat{x}$, а $\hat{x}$ – найденное приближенное решение. В данной работе использовалось значение $tol_{rel} = 1e-8$. Количество итераций является ограниченным, в данной работе был установлен предел, равный 10000 итерациям.

Производилось четыре вида тестов:
\begin{enumerate}
	\item на системе с известным точным решением $x$ :
	\item на 500 системах порядка 4 с матрицами Стилтьеса;
	\item на 250 системах порядка 9 с разреженными матрицами Стилтьеса;
	\item на 1300 системах порядка от 4 до 9 с матрицами Стилтьеса.
\end{enumerate}

Для первого теста использовалась следующая система:

\begin{equation} \label{sys:1}
	A = \begin{pmatrix}
		10 & -1  & 2  & 0  \\
		-1 &  11 & -1 & 3  \\
		2  & -1  & 10 & -1 \\
		0  & 3   & -1 & 8  \\
	\end{pmatrix}
	x = \begin{pmatrix}
		1 \\ 2 \\ -1 \\ 1
	\end{pmatrix}
	b = \begin{pmatrix}
		6 \\ 25 \\ -11 \\ 15
	\end{pmatrix}
\end{equation}
тогда как в остальных случаях использовались случайным образом сгенерированные матрицы. Размерность систем была ограничена высокой вычислительной сложностью выбранного способа генерации матриц Стилтьеса: матрицы сначала конструировались специальным образом, затем среди них производился поиск положительно определённых (с помощью разложения Холесского). 

Можно отметить, что метод SOR при параметре релаксации $w=1.0$ выполняет ровно те же самые действия, что и метод Гаусса-Зейделя, чем и обусловлены соответствующие результаты. 

\begin{figure}[ht]
	\begin{center}
   		\begin{tabular}{ | l | l | p{5cm} |}
    	\hline
    	Используемый метод   & Количество итераций  \\ \hline
    	Метод Якоби          & 26  \\ \hline
    	Метод Гаусса-Зейделя & 9  \\ \hline
    	Метод SOR, $w=1.0$   & 9  \\ \hline
    	Метод SSOR, $w=1.0$  & 11  \\ \hline
    	\end{tabular}
	\end{center}
	\captionof{table}{Количество итераций, требуемое для решения системы \ref{sys:1}}
\end{figure}


Так как среднее значение количества итераций, необходимых для решения случайной системы, сгенерированных частным образом, не является достаточно информативной метрикой, дополнительные сведения предоставляют графики 4 и 5. На них демонстрируется зависимость между количеством итераций, требуемых для решения системы, и числом обусловленности матрицы данной системы.

\begin{figure}[h]
	\begin{center}
   		\begin{tabular}{ | l | l | p{5cm} |}
    	\hline
    	Используемый метод   & Количество итераций  \\ \hline
    	Метод Якоби          & 378.67  \\ \hline
    	Метод Гаусса-Зейделя & 253.25  \\ \hline

   		Метод SOR,  $w=0.25$ & 709.28  \\ \hline
    	Метод SOR,  $w=1.0$  & 253.25  \\ \hline
    	Метод SOR,  $w=1.5$  & 125.88  \\ \hline

    	Метод SSOR, $w=0.25$ & 709.43  \\ \hline
    	Метод SSOR, $w=1.0$  & 273.68  \\ \hline
    	Метод SSOR, $w=1.5$  & 190.17  \\ \hline

    	\end{tabular}
	\end{center}
	\captionof{table}{Среднее количество итераций, необходимых для решения системы порядка 4 с матрицей Стилтьеса (в результате решения 500 случайных систем)}
\end{figure}

\begin{figure}[h]
	\begin{center}
   		\begin{tabular}{ | l | l | p{5cm} |}
    	\hline
    	Используемый метод   & Количество итераций  \\ \hline
    	Метод Якоби          & 306.08  \\ \hline
    	Метод Гаусса-Зейделя & 209.52  \\ \hline

   		Метод SOR,  $w=0.25$ & 577.48  \\ \hline
    	Метод SOR,  $w=1.0$  & 209.52  \\ \hline
    	Метод SOR,  $w=1.5$  & 87.24   \\ \hline

    	Метод SSOR, $w=0.25$ & 571.48  \\ \hline
    	Метод SSOR, $w=1.0$  & 219.28  \\ \hline
    	Метод SSOR, $w=1.5$  & 180.04  \\ \hline

    	\end{tabular}
	\end{center}
	\captionof{table}{Среднее количество итераций, необходимых для решения системы порядка 9 с матрицей Стилтьеса (в результате решения 250 случайных разреженных систем)}
\end{figure}



\begin{figure}[ht]
\includegraphics[width=\textwidth,height=\textheight,keepaspectratio]{im2_figure_3}
\caption{График зависимости количества итераций от числа обусловленности матрицы системы для методов Якоби, Гаусса-Зейделя и SOR. Ось X – число обусловленности, ось Y - количество итераций. }

\end{figure}


\begin{figure}[ht]
\includegraphics[width=\textwidth,height=\textheight,keepaspectratio]{im2_figure_4}
\caption{График зависимости количества итераций от числа обусловленности матрицы системы для метода SOR при различных значениях параметра релаксации. Ось X – число обусловленности, ось Y - количество итераций. }
\end{figure}

% Got 751 dense and 550 sparse systems
% jacobi 1561.28621397
% ssor_method 1118.08893506
% gauss_seidel 1047.52090785


