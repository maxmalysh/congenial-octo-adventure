\section{Теоретические сведения}

%%%%%%%%%%%%%%%%%%%%%%%%%%%%%%%%%%%%%%%%%%%%%%%%%%%%%%%%%%%%%%%%%
\subsection{Матрицы Стилтьеса}

Монотонная матрица – матрица, обратная к которой имеет неотрицательные коэффициенты.

M-матрица – монотонная матрица с неположительными внедиагональными элементами.

Матрица Стильтьеса – симметрическая M-матрица.

Матрица Стильтьеса является положительно определенной.

\subsection{Метод сопряжённых градиентов}

Идея обуславливания итерационного процесса решения $Ax=b$ состоит в выборе невырожденной матрицы $S$ и переходу к системе $\hat A\hat x=\hat b$, где $\hat A=SAS^\ast, \hat b=Sb, S^\ast\hat x=x$, для которой итерационный процесс будет иметь более высокую скорость сходимости.

\subsection{Стационарные итерационные процессы}
Якоби, Гаусс-Зейдель, SOR, SSOR