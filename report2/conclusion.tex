\section{Выводы и результаты}
%%%%%%%%%%%%%%%%%%%%%%%%%%%%%%%%%%%%%

\begin{figure}[ht]
%\includegraphics[width=\textwidth,height=\textheight,keepaspectratio]{file_name_without_extension}
\caption{Подпись к графику}
\end{figure}



Стационарные итерационные процессы, используемые для решения систем вида $Ax=b$, исследовались на предмет скорости сходимости. Относительная ошибка находилась в виде отношения $||r|| / ||b||$ нормы вектора невязки $r = b - A\hat{x}$ к норме вектора $b$.  
Метод Якоби для системы порядка 4 сошёлся к результату, имеющему относительную ошибку, равную 2.322e-06, за 15 итераций. 
Для метода Гаусса-Зейделя потребовалось 9 итераций 7.615e-10. 
Методы последовательной релаксации SOR и его симметричный вариант SSOR при параметре релаксации $w=1.0$ показали результаты, аналогичные методу Гаусса-Зейделя. При других значения ($w \in (0, 2.0)$) количество итераций увеличивалось, а точность падала.




