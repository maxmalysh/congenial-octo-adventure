\section{Теоретическая составляющая}

%%%%%%%%%%%%%%%%%%%%%%%%%%%%%%%%%%%%%%%%%%%%%%%%%%%%%%%%%%%%%%%%%
\subsection{LU-разложения}
\subsubsection{PLU-, LUP- и PLUP-' разложения}
\subsubsection{Обычное и модифицированное разложение Брюа}


Разложением Брюа элемента $A$ группы $Gl_n(\mathbb{k}), \mathbb{k} = \mathbb{R}, \mathbb{C}$ является его представление в виде $A=LPL'$, где $L$ и $L'$ являются невырожденными нижними треугольными матрицами, а $P$ – матрица-перестановка. 

Модифицированным разложением Брюа матрицы $A$ называется разложение $A=LPU$, где $L$ и $U$ – невырожденные нижняя и верхняя треугольные матрицы, а $P$ – матрица-перестановка.

Между разложением Брюа и его модифицированной версией существует тесная взаимосвязь. Если известно модифицированное разложение Брюа матрицы $AQ=LPU$, где $Q$ – матрица перестановка вида
\[
Q = 
	\begin{pmatrix}
	0 & 0 & \dots & 0 & 1  \\
	0 & 0 & \dots & 1 & 0  \\
	\hdotsfor{5}           \\
	0 & 1 & \dots & 0 & 0  \\
	1 & 0 & \dots & 0 & 0  
	\end{pmatrix},
\]
тогда разложение Брюа матрицы $A$ можно определить как $A=L(PQ)L'$, полагая $L'=QUQ$. Сходным образом на основе разложение Брюа матрицы $AQ$ можно построить модифицированное разложение Брюа матрицы $A$.


%%%%%%%%%%%%%%%%%%%%%%%%%%%%%%%%%%%%%%%%%%%%%%%%%%%%%%%%%%%%%%%%%
\subsection{Метод Гаусса}
\subsubsection{Связь с LU-разложением}

Решение линейной системы Ax=b с использованием метода Гаусса с выбором по столбцу сводится к решению двух треугольных систем

\[
\begin{cases}
     Ly=Pb   \\
     Ux=y    
\end{cases}
\]
где $PA=LU$ (в стандартной реализации без поиска самого разложения мы решаем на заключительном этапе систему $Ux=L^{-1}Pb$). Ход решения системы $Ly=Pb$ называют прямым ходом (англ. \textit{forward substitution}), а решение $Ux=y$ – обратной подстановкой (англ. \textit{backward substitution}).
Решение с использованием метода Гаусса с выбором по строке соответствует решению системы

\[
\begin{cases}
	Ly = b \\
	Uz = y \\
	x = Pz,
\end{cases}
\] 
где $AP=LU$ (без нахождения разложения мы решаем систему $Uz=L^{-1}b и находим x=Pz$), и, наконец, решению с использованием метода Гаусса с полным набором отвечает решение системы 

\[
\begin{cases}
	Ly = Pb \\
	Uz = y  \\
	x = P'z,
\end{cases}
\] 
где $PAP'=LU$ (здесь на заключительном этапе мы решаем систему $Uz = L^{-1}b$, $x=P'z$). 


%%%%%%%%%%%%%%%%%%%%%%%%%%%%%%%%%%%%%%%%%%%%%%%%%%%%%%%%%%%%%%%%%
\subsubsection{Оптимальное заполнение для разреженных матриц}

Lorem ipsum 


%%%%%%%%%%%%%%%%%%%%%%%%%%%%%%%%%%%%%%%%%%%%%%%%%%%%%%%%%%%%%%%%%
\subsection{Итерационное уточнение простой и переменной точности}

Один из приёмов улучшения качества решения состоит в использовании итерационного уточнения, которое выполняется следующим образом:

\begin{enumerate}
	\item исходная система $Ax=B$ решается с помощью какого-либо алгоритма, в результате чего вычисляется решение $\hat{x}$ ;
	\item тот же самый алгоритм применяется к системе $Az=r$, где $r$ – вектор невязки $r=b-A\hat{x}$;
	\item вычисляется вектор $\hat{x}_{new} = \hat{x} + \hat{z}$, являющийся уточнённым решением исходной системы
\end{enumerate}

В точной арифметике решение осталось бы неизменным, так как в этом случае $r = z = 0$. В арифметике с плавающей запятой такого не происходит, но, тем не менее, найденный вектор $\hat{x}_{new}$ может оказаться не сильно лучше вектора $\hat{x}$, если тот был уже близок к решению. В ином случае несколько шагов такого уточнения могут значительно улучшить качество решения (т.е. количество верных разрядов) $\hat{x}$.

Значительно более эффективной версией процесса итерационного уточнения является версия этого алгоритма, в которой вектор невязки вычисляется с удвоенной точностью. 


%%%%%%%%%%%%%%%%%%%%%%%%%%%%%%%%%%%%%%%%%%%%%%%%%%%%%%%%%%%%%%%%%
\subsection{Оценщик числа обусловленности матрицы системы в строчной норме}

Алгоритм обратной подстановки для верхней треугольной системы $Tx=y$ может быть переписан в следующем виде:
\begin{enumerate}
	\item на исходном этапе вспомогательный вектор-столбец $p(n)=(p_1, \dots, p_n)^t$ полагается равным нулю;
	\item затем для всех $k=n, \dots, 1$ вычисляются
	\begin{enumerate}
		\item $x_k =(y_k - p_k) / t_{kk}$ 
		\item $p(k-1) = p(k-1) + x_{k}t_{k}(k-1)$, \\ где 
			\begin{tabular}{c}
		 		$p(k-1)=(p_1, \dots, p_{k-1})^t$, \\
		 		$t_k(k-1)=(t_{1k}, \dots, t_{k-1k})^t$ \\
			\end{tabular}
		\end{enumerate}
\end{enumerate}

Алгоритм оценки строится следующим образом:
\begin{enumerate}
	\item На шаге $ k = n, \dots, 1$ выбирается $y_k \in \{\pm1\}$ из следующих соображений:
	\item вычисляются \\ \\
		\begin{tabular}{l r}
			$x^{+}_{k} = (1-p_k)/t_{kk}$, &	
			$s^{+}_{k} = |x^{+}_{k}| + || p(k-1) + x^{+}_{k}t_{k}(k-1) ||_1$, \\
			
			$x^{-}_{k} = (-1-p_k)/t_{kk}$, &	
			$s^{-}_{k} = |x^{-}_{k}| + || p(k-1) + x^{-}_{k}t_{k}(k-1) ||_1$, \\
		\end{tabular}
	\item затем при $s^{+}_{k} \geq s^{-}_{k}$ мы полагаем $x_{k} = x_{k}^{+}$, иначе – $x_{k} = x^{-}_{k}$
	\item Норма $||x||_{ \infty }$ найденного в итоге вектора $x$ объявляется приближённым значением $||T^{-1}||_{ \infty }$.
\end{enumerate}

В соответствии с этим алгоритмом предлагается использовать следующий алгоритм оценки $k_{\infty}(A)$ при известном разложении $PA=LR$, найденном в методе Гаусса с частичным выбором по столбцу. Итак,
\begin{enumerate}
	\item применим описанный выше алгоритм к системе $R^{t}x=y$ и найдём соответствующие вектора $\hat{x}$ и $\hat{y}$;
	\item решим системы $L^{t}u=\hat{x}$, $Lw=Pu$, $Rz=w$ ($A^{t}P^{t}u=R^{t}L^{t}u = \hat{y}$, $LRz=Pu$, $Az=u$);
	\item положим $\hat{ k_{\infty} (A) } = ||A||_{\infty} ||z||_{\infty} / ||u||_{\infty}$.
\end{enumerate}



