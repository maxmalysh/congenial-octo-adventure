\section{Теоретическая составляющая}

%%%%%%%%%%%%%%%%%%%%%%%%%%%%%%%%%%%%%%%%%%%%%%%%%%%%%%%%%%%%%%%%%
\subsection{LU-разложения}
\subsubsection{PLU-, LUP- и PLUP-' разложения}
\subsubsection{Обычное и модифицированное разложение Брюа}


Разложением Брюа элемента $A$ группы $Gl_n(\mathbb{k}), \mathbb{k} = \mathbb{R}, \mathbb{C}$ является его представление в виде $A=LPL'$, где $L$ и $L'$ являются невырожденными нижними треугольными матрицами, а $P$ – матрица-перестановка. 

Модифицированным разложением Брюа матрицы $A$ называется разложение $A=LPU$, где $L$ и $U$ – невырожденные нижняя и верхняя треугольные матрицы, а $P$ – матрица-перестановка.

Между разложением Брюа и его модифицированной версией существует тесная взаимосвязь. Если известно модифицированное разложение Брюа матрицы $AQ=LPU$, где $Q$ – матрица перестановка вида
\[
Q = 
	\begin{pmatrix}
	0 & 0 & \dots & 0 & 1  \\
	0 & 0 & \dots & 1 & 0  \\
	\hdotsfor{5}           \\
	0 & 1 & \dots & 0 & 0  \\
	1 & 0 & \dots & 0 & 0  
	\end{pmatrix},
\]
тогда разложение Брюа матрицы $A$ можно определить как $A=L(PQ)L'$, полагая $L'=QUQ$. Сходным образом на основе разложение Брюа матрицы $AQ$ можно построить модифицированное разложение Брюа матрицы $A$.


%%%%%%%%%%%%%%%%%%%%%%%%%%%%%%%%%%%%%%%%%%%%%%%%%%%%%%%%%%%%%%%%%
\subsection{Метод Гаусса}
\subsubsection{Связь с LU-разложением}

Решение линейной системы Ax=b с использованием метода Гаусса с выбором по столбцу сводится к решению двух треугольных систем

\[
\begin{cases}
     Ly=Pb   \\
     Ux=y    
\end{cases}
\]
где $PA=LU$ (в стандартной реализации без поиска самого разложения мы решаем на заключительном этапе систему $Ux=L^{-1}Pb$). Ход решения системы $Ly=Pb$ называют прямым ходом (англ. \textit{forward substitution}), а решение $Ux=y$ – обратным ходом (англ. \textit{backward substitution}).
Решение с использованием метода Гаусса с выбором по строке соответствует решению системы

\[
\begin{cases}
	Ly = b \\
	Uz = y \\
	x = Pz,
\end{cases}
\] 
где $AP=LU$ (без нахождения разложения мы решаем систему $Uz=L^{-1}b и находим x=Pz$), и, наконец, решению с использованием метода Гаусса с полным набором отвечает решение системы 

\[
\begin{cases}
	Ly = Pb \\
	Uz = y  \\
	x = P'z,
\end{cases}
\] 
где $PAP'=LU$ (здесь на заключительном этапе мы решаем систему $Uz = L^{-1}b$, $x=P'z$). 


%%%%%%%%%%%%%%%%%%%%%%%%%%%%%%%%%%%%%%%%%%%%%%%%%%%%%%%%%%%%%%%%%
\subsubsection{Оптимальное заполнение для разреженных матриц}

Lorem ipsum 


%%%%%%%%%%%%%%%%%%%%%%%%%%%%%%%%%%%%%%%%%%%%%%%%%%%%%%%%%%%%%%%%%
\subsection{Итерационное уточнение простой и переменной точности}

Один из приёмов улучшения качества решения состоит в использовании итерационного уточнения, которое выполняется следующим образом:

\begin{enumerate}
	\item исходная система $Ax=B$ решается с помощью какого-либо алгоритма, в результате чего вычисляется решение $\hat{x}$ ;
	\item тот же самый алгоритм применяется к системе $Az=r$, где $r$ – вектор невязки $r=b-A\hat{x}$;
	\item вычисляется вектор $\hat{x}_{new} = \hat{x} + \hat{z}$, являющийся уточнённым решением исходной системы
\end{enumerate}


%%%%%%%%%%%%%%%%%%%%%%%%%%%%%%%%%%%%%%%%%%%%%%%%%%%%%%%%%%%%%%%%%
\subsection{Оценщик числа обусловленности матрицы системы в строчной норме}

Lorem ipsum

