\section{Теоретическая составляющая}

%%%%%%%%%%%%%%%%%%%%%%%%%%%%%%%%%%%%%%%%%%%%%%%%%%%%%%%%%%%%%%%%%
\subsection{$LU$-разложение и его варианты}
\subsubsection{$PLU$-, $LUP$- и $PLUP'$- разложения}

Элементарные преобразования, осуществляемые над строками матрицы на каждом шаге метода Гаусса, имеют интерпретацию в терминах матричных умножений. Если учесть то, какой вид имеют матрицы обратных элементарных преобразований, то становится понятно, что произведение этих матриц будет нижнетреугольной матрицей $L$. В то же время после прямого хода метода Гаусса матрица будет преобразована в верхнюю унитреугольную матрицу $U$.

Для построения $LU$-разложения можно повторить все шаги метода Гаусса с той лишь разницей, что на каждом шаге нам следует сохранить компоненты первого столбца ведущей квадратной подматрицы на их местах (поскольку матрица $U$ унитреугольна, её единичные диагональные составляющие нас не интересуют). Иначе говоря, мы можем модифицировать метод Гаусса с целью нахождения $LU$ разложения следующим образом. На первом шаге мы переходим от матрицы $A = A^{(0)}$ к матрице $A'^{(1)}$ вида

\[
A'^{ (1) } = 
	\begin{pmatrix}
		a_{11}^{ (0) } & c_{12}			& \hdots & c_{1n}			\\
		a_{21}^{ (0) } & a_{22}^{ (1) } & \hdots & a_{2n}^{ (1) }	\\
		\vdots		   & \vdots			& \ddots & \vdots			\\
		a_{n1}^{ (0) } & a_{n2}^{ (1) } & \hdots & a_{nn}^{ (1) } 	
	\end{pmatrix},
\]
а на шаге $k$ должны получить матрицу $A'^{ (k) }$, 

\[
A'^{ (k) } = 
	\begin{pmatrix}
		
		a_{11}^{ (0) }   & c_{12}			& \hdots & c_{1k}             & c_{1k+1}   & \hdots & c_{1n}                 \\
		\hdots           & \hdots           & \hdots & \hdots             & \hdots     & \hdots & \hdots                 \\
		a_{k-11}^{ (0) } & a_{k-12}^{ (1) } & \hdots & c_{k-1k}           & c_{k-1k+1} & \hdots & c_{k-1n}               \\
		a_{ k1 }^{ (0) } & a_{k2}^{ (1) }   & \hdots & a_{kk}^{ (k-1) }   & c_{kk+1}   & \hdots & c_{kn}                 \\
		a_{k+11}^{ (0) } & a_{k+12}^{ (1) } & \hdots & a_{k+1k}^{ (k-1) } & a_{k+1k+1}^{ (k) } & \hdots & a_{k+1n}^{(k)} \\
		\hdots           & \hdots           & \hdots & \hdots             & \hdots     & \hdots & \hdots                 \\
		a_{n1}^{ (0) }  & a_{n2}^{ (1) }    & \hdots & a_{nk}^{ (k-1) }   & a_{ nk+1 }^{ (k) } & \hdots & a_{nn}^{ (k) }       
		
	\end{pmatrix},
\]
где все указанные здесь компоненты такие же, как и в методе Гаусса. В итоге после выполнения $n-2$ шагов мы получим матрицу $A'^{(n)}$

\[
A'^{ (1) } = 
	\begin{pmatrix}
		a_{11}^{ (0) } & c_{12}			& \hdots & c_{1n}			\\
		a_{21}^{ (0) } & a_{22}^{ (1) } & \hdots & a_{2n}			\\
		\vdots		   & \vdots			& \ddots & \vdots			\\
		a_{n1}^{ (0) } & a_{n2}^{ (1) } & \hdots & a_{nn}^{ (n-1) } 	
	\end{pmatrix},
\] 
в которой верхний треугольник выше главной диагонали заполнен компонентами $U$-составляющей матрицы $A$, а нижний треугольник и диагональ – компонентами её $L$-составляющей. Сложность такого алгоритма составляет $2/3n^3 + O(n^2)$.


Для любой невырожденной  матрицы $A$ можно подобрать матрицы перестановки $P$, $P'$ или $P_1$ и $P_2$ такие, что матрицы $PA$, $AP'$ и $P_1 A P_2$ обладают $LU$-разложением. Соответственно, существуют модернизации метода Гаусса, использующие три основных стратегии выбора ведущего элемента на $k$-ом шаге алгоритма: стратегия частичного выбора по первым строке или столбцу ведущей подматрицы, и стратегия полного выбора по всей ведущей подматрице. Идея этих модернизаций состоит в следующем изменении процедуры вычисления матрицы $A^{(k)}$ на $k$-ом шаге, $1 \leq k \leq n - 1$:

\begin{enumerate}
	\item среди элементов первого столбца $a^{(k-1)}_{ik}$, $i = k, \hdots, n$, ведущей подматрицы находится элемент $a^{(k-1)}_{i_k k}$ с наибольшим модулем, после чего осуществляется переименование: $k$-ая строка матрицы $A^{ (k-1) }$ становится $i_k$-ой, а $i_k$-ая строка – $k$-ой (т.е. осуществляется перестановка местами $k$-ой и $i_k$-ой строк матрицы  $A^{ (k-1) }$ ), а затем выполняется обычный ($k$-ый) шаг метода Гаусса (стратегия частичного выбора по столбцу);
	\item среди элементов первой строки $a^{(k-1)}_{kj}$, $j = k, \hdots, n$, ведущей подматрицы находится элемент $a^{(k-1)}_{k j_k}$ с наибольшим модулем, после чего осуществляется перенумерация переменных (столбцов): $x_k$ и $x_{j_k}$ меняются местами (т.е. осуществляется перестановка местами $k$-го и $j_k$-го столбцов матрицы $A^{ (k-1) }$), вслед за этим выполняется $k$-ый шаг метода Гаусса в привычной форме (стратегия частичного выбора по строке);
	\item среди всех элементов $a^{(k-1)}_{ij}$, $i, j = k, \hdots, n$, ведущей подматрицы находится элемент $a^{(k-1)}_{i_k j_k}$ с наибольшим модулем и осуществляется перестановка местами строк с номерами $k$ и $i_k$ и столбцов с номерами $k$ и $j_k$ матрицы $A^{ (k-1) }$, после чего выполняется обычный $k$-ый шаг метода Гаусса (стратегия полного выбора по всей матрице).
\end{enumerate}


\subsubsection{Обычное и модифицированное разложение Брюа}

Разложением Брюа элемента $A$ группы $Gl_n(\mathbb{k}), \mathbb{k} = \mathbb{R}, \mathbb{C}$ является его представление в виде $A=LPL'$, где $L$ и $L'$ являются невырожденными нижними треугольными матрицами, а $P$ – матрица-перестановка. 

Модифицированным разложением Брюа матрицы $A$ называется разложение $A=LPU$, где $L$ и $U$ – невырожденные нижняя и верхняя треугольные матрицы, а $P$ – матрица-перестановка.

Между разложением Брюа и его модифицированной версией существует тесная взаимосвязь. Если известно модифицированное разложение Брюа матрицы $AQ=LPU$, где $Q$ – матрица перестановка вида
\[
Q = 
	\begin{pmatrix}
	0 & 0 & \dots & 0 & 1  \\
	0 & 0 & \dots & 1 & 0  \\
	\hdotsfor{5}           \\
	0 & 1 & \dots & 0 & 0  \\
	1 & 0 & \dots & 0 & 0  
	\end{pmatrix},
\]
тогда разложение Брюа матрицы $A$ можно определить как $A=L(PQ)L'$, полагая $L'=QUQ$. Сходным образом на основе разложение Брюа матрицы $AQ$ можно построить модифицированное разложение Брюа матрицы $A$.

Опишем алгоритм построения разложения $A=LPU$, позволяющий построить данное разложение в результате выполнения $n$ шагов, где $n$ – размерность исходной матрицы $A^{ (0) } = A = (a_{ij}) \in Gl_{n}(\mathbb{k})$, последовательность матриц $A^{ (i) } = L_{i} A^{(i-1)} U_{i}$, $i=1, \hdots, n$, в которой матрица $A^{ (i) }$ получается из своей предшественницы $A^{ (i-1) }$ домножением слева и справа на подходящие невырожденные нижнюю и верхнюю треугольные матрицы $L_i$ и $U_i$, а последняя матрица $A^{(n)}$ является матрицей-перестановкой. Требуемое разложение определяется равенствами $ P = A^{ (n) }$ , $ L = (L_n \dotsb L_1)^{-1} = L_1^{-1} \dotsb L_n^{-1}$, $U = (U_1 \dotsb U_n)^{-1} = U_n^{-1} \dotsb U_1^{-1}$.

Опишем типичный $k$-ый шаг этого процесса, соответствующий переходу от матрицы $A^{ (k-1) }$ к матрице $A^{ (k) } = L_k A^{ (k-1) } U_k$ и построению матриц $L_k$ и $U_k$. Он состоит в следующем: в $k$-ой строке матрицы $A^{ (k-1) }$ находится первый ненулевой элемент $a^{ (k-1) }_{k i_k}$ (т.е. $ a^{ (k-1) }_{ k i } = 0 $ при всех $1 \leq i \leq i_k$), затем осуществляется масштабирование $k$-ой строки матрицы $A^{ (k-1) }$ делением её элементов на $a_{k i_k}$ и вычитание из каждой строки с номером $s=k+1, \hdots, n$ и каждого столбца с номером $ t = i_k + 1, \hdots, n $, $k$-ой строки и $i_k$-го столбца масштабированной матрицы $A^{ (k-1)' }$ домноженных на элементы $ a^{ (k-1)' }_{ s i_k } = a^{ (k-1) }_{ s i_k} $ и $ a_{kt} (k-1)' = a^{(k-1)}_{kt} / a^{(k-1)}_{k i_k}$ с целью обнуления всех компонент $i_k$-го столбца, начиная с $k+1$-ой, и всех компонент $k$-строки, начиная с $i_k + 1$-ой. В терминах матричных умножений эти действия реализуются в виде перехода от $A^{ (k-1) }$ к $A^{ (k) } = L_k A^{ (k-1) } U_k$, где матрицы $L_k$ и $U_k$ отличаются от единичной только своими $k$-ым столбцом и $i_k$-ой строкой, соответственно,

\[
L_k = 
\begin{pmatrix}
	1 & 0 & \hdots & 0 & 0 & \hdots & 0 \\
	0 & 1 & \hdots & 0 & 0 & \hdots & 0 \\
	\hdots & \hdots & \hdots & \hdots & \hdots & \hdots & \hdots \\
	0 & 0 & \hdots & 1 / a^{(k-1)}_{k i_k} & 0 & \hdots & 0 \\
	0 & 0 & \hdots & -a^{(k-1)}_{k+1i_k}/a^{(k-1)}_{k i_k} & 1 & \hdots & 0 \\
	\hdots & \hdots & \hdots & \hdots & \hdots & \hdots & \hdots  \\
	0 & 0 & \hdots & -a^{(k-1)}_{n i_k} / a^{(k-1)}_{k i_k} & 0 & \hdots & 1 \\
\end{pmatrix},
\]

\[
U_k = 
\begin{pmatrix}
	1 & 0 &	\hdots & 0 & 0 & \hdots	& 0			\\
	0 & 1 & \hdots & 0 & 0 & \hdots	& 0				\\
	\hdots & \hdots & \hdots & \hdots & \hdots & \hdots & \hdots \\
	0 & 0 & \hdots & 1 & -a^{(k-1)}_{k i_{k}+1} / a^{(k-1)}_{k i_k} & \hdots & 
						 -a^{(k-1)}_{k n} / a^{(k-1)}_{k i_k}		\\
	0 & 0 & \hdots & 0 & 1 & \hdots	& 0					\\
	\hdots & \hdots & \hdots & \hdots & \hdots & \hdots & \hdots \\
	0 & 0 & \hdots & 0 & 0 & \hdots & 1					\\		
\end{pmatrix}.
\]
			
%%%%%%%%%%%%%%%%%%%%%%%%%%%%%%%%%%%%%%%%%%%%%%%%%%%%%%%%%%%%%%%%%
\subsection{Метод Гаусса}
\subsubsection{Связь с LU-разложением}

Решение линейной системы Ax=b с использованием метода Гаусса с выбором по столбцу сводится к решению двух треугольных систем

\[
\begin{cases}
     Ly=Pb   \\
     Ux=y    
\end{cases}
\]
где $PA=LU$ (в стандартной реализации без поиска самого разложения мы решаем на заключительном этапе систему $Ux=L^{-1}Pb$). Ход решения системы $Ly=Pb$ называют прямым ходом (англ. \textit{forward substitution}), а решение $Ux=y$ – обратной подстановкой (англ. \textit{backward substitution}).
Решение с использованием метода Гаусса с выбором по строке соответствует решению системы

\[
\begin{cases}
	Ly = b \\
	Uz = y \\
	x = Pz,
\end{cases}
\] 
где $AP=LU$ (без нахождения разложения мы решаем систему $Uz=L^{-1}b и находим x=Pz$), и, наконец, решению с использованием метода Гаусса с полным набором отвечает решение системы 

\[
\begin{cases}
	Ly = Pb \\
	Uz = y  \\
	x = P'z,
\end{cases}
\] 
где $PAP'=LU$ (здесь на заключительном этапе мы решаем систему $Uz = L^{-1}b$, $x=P'z$). 


%%%%%%%%%%%%%%%%%%%%%%%%%%%%%%%%%%%%%%%%%%%%%%%%%%%%%%%%%%%%%%%%%
\subsubsection{Оптимальное заполнение для разреженных матриц}

Под локальным заполнением на $k$-ом шаге метода Гаусса понимается количество ненулевых элементов матрицы $A^{(k-1)}$, которые стали ненулевыми после выполнения данного шага. Интерес представляет не сама матрица $A^{(k-1)}$, а её нижний блок $(n-k)\times(n-k)$, преобразуемый в новую ведущую подматрицу. 

Обозначим через $B_k$ матрицу, полученную из матрицы $(a_{ij}^{(k-1)})^{n}_{i,j=k}$ заменой ненулевых элементов единицами, и через $\overline{B_k}$ – матрицу, полученную заменой единичных элементов матрицы $B_k$ нулями, а нулей – единицами, $\overline{B_k}=M - B_k$, где $M$ – матрица размера $(n-k+1)\times(n-k+1)$, состоящая из одних единиц. 

Теорема Тьарсона: если элемент $a^{(k-1)}_{i+k-1j+k-1} \ne 0$, $i,j=1, n-k+1$, выбирается в качестве ведущего на $k$-ом шаге метода Гаусса, тогда локальное заполнение на этом шаге совпадает с $(i+k-1, j+k-1)$-ым элементом матрицы $G_k = B_k \overline{B_{k}^{t}} B_k = ( g^{(k)}_{st} )^{n}_{s,t=k}$.

Можно модифицировать метод Гаусса в целях более оптимального заполнения разреженных матриц следующим образом. Зафиксируем пороговое $\epsilon > 0$. Выберем в качестве ведущего элемента на $k$-ом шаге элемент $a^{(k-1)}_{i+k-1 j+k-1}$, $i, j = 1, \hdots, n-k+1$, для которого элемент $g^{(k)}_{ij}$ является минимальным среди всех элементов $ |a^{(k-1)}_{i+k-1 j+k-1} | > \epsilon $ (если таких элементов несколько, то мы выберем среди них элемент с наибольшим модулем). Затем выполним шаг метода Гаусса. 

Полученный вариант метода Гаусса с выбором ведущего элемента обеспечивает оптимальное заполнение среди всех возможных вариантов выбора ведущего элемента, по модулю большего $\epsilon$, хотя и имеет весьма значительную вычислительную сложность. Он также соответствует факторизации $PAP'=LU$ как и описанный ранее метод Гаусса с полным выбором. 

%%%%%%%%%%%%%%%%%%%%%%%%%%%%%%%%%%%%%%%%%%%%%%%%%%%%%%%%%%%%%%%%%
\subsection{Итерационное уточнение простой и переменной точности}

Один из приёмов улучшения качества решения состоит в использовании итерационного уточнения, которое выполняется следующим образом:

\begin{enumerate}
	\item исходная система $Ax=B$ решается с помощью какого-либо алгоритма, в результате чего вычисляется решение $\hat{x}$ ;
	\item тот же самый алгоритм применяется к системе $Az=r$, где $r$ – вектор невязки $r=b-A\hat{x}$;
	\item вычисляется вектор $\hat{x}_{new} = \hat{x} + \hat{z}$, являющийся уточнённым решением исходной системы
\end{enumerate}

В точной арифметике решение осталось бы неизменным, так как в этом случае $r = z = 0$. В арифметике с плавающей запятой такого не происходит, но, тем не менее, найденный вектор $\hat{x}_{new}$ может оказаться не сильно лучше вектора $\hat{x}$, если тот был уже близок к решению. В ином случае несколько шагов такого уточнения могут значительно улучшить качество решения (т.е. количество верных разрядов) $\hat{x}$.

Значительно более эффективной версией процесса итерационного уточнения является версия этого алгоритма, в которой вектор невязки вычисляется с удвоенной точностью. 


%%%%%%%%%%%%%%%%%%%%%%%%%%%%%%%%%%%%%%%%%%%%%%%%%%%%%%%%%%%%%%%%%
\subsection{Оценщик числа обусловленности матрицы системы в строчной норме}

Алгоритм обратной подстановки для верхней треугольной системы $Tx=y$ может быть переписан в следующем виде:
\begin{enumerate}
	\item на исходном этапе вспомогательный вектор-столбец $p(n)=(p_1, \dots, p_n)^t$ полагается равным нулю;
	\item затем для всех $k=n, \dots, 1$ вычисляются
	\begin{enumerate}
		\item $x_k =(y_k - p_k) / t_{kk}$ 
		\item $p(k-1) = p(k-1) + x_{k}t_{k}(k-1)$, \\ где 
			\begin{tabular}{c}
		 		$p(k-1)=(p_1, \dots, p_{k-1})^t$, \\
		 		$t_k(k-1)=(t_{1k}, \dots, t_{k-1k})^t$ \\
			\end{tabular}
		\end{enumerate}
\end{enumerate}

Алгоритм оценки строится следующим образом:
\begin{enumerate}
	\item На шаге $ k = n, \dots, 1$ выбирается $y_k \in \{\pm1\}$ и вычисляются:\\\\
		\begin{tabular}{l r}
			$x^{+}_{k} = (1-p_k)/t_{kk}$, &	
			$s^{+}_{k} = |x^{+}_{k}| + || p(k-1) + x^{+}_{k}t_{k}(k-1) ||_1$, \\
			
			$x^{-}_{k} = (-1-p_k)/t_{kk}$, &	
			$s^{-}_{k} = |x^{-}_{k}| + || p(k-1) + x^{-}_{k}t_{k}(k-1) ||_1$, \\
		\end{tabular}
	\item Затем при $s^{+}_{k} \geq s^{-}_{k}$ мы полагаем $x_{k} = x_{k}^{+}$, иначе – $x_{k} = x^{-}_{k}$
	\item Норма $||x||_{ \infty }$ найденного в итоге вектора $x$ объявляется приближённым значением $||T^{-1}||_{ \infty }$.
\end{enumerate}

В соответствии с этим алгоритмом предлагается использовать следующий алгоритм оценки $k_{\infty}(A)$ при известном разложении $PA=LR$, найденном в методе Гаусса с частичным выбором по столбцу. Итак,
\begin{enumerate}
	\item применим описанный выше алгоритм к системе $R^{t}x=y$ и найдём соответствующие вектора $\hat{x}$ и $\hat{y}$;
	\item решим системы $L^{t}u=\hat{x}$, $Lw=Pu$, $Rz=w$ ($A^{t}P^{t}u=R^{t}L^{t}u = \hat{y}$, $LRz=Pu$, $Az=u$);
	\item положим $\hat{ k_{\infty} (A) } = ||A||_{\infty} ||z||_{\infty} / ||u||_{\infty}$.
\end{enumerate}



