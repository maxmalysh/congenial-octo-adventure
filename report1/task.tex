\section{Постановка задачи}
\begin{enumerate}
	\item реализовать алгоритмы построения разложений произвольной невырожденной матрицы $A$ вида $A=P_1 L_1 U_1$, $L_2 U_2 P_2$ и $A=P_3 L_3 U_3 P_3'$ из методов Гаусса с выбором по строке, столбцу и всей матрицы, а также алгоритмы построения для неё разложения Брюа $A=L_4 P_4 L_4$ и его модифицированной версии $A=L_5 P_5 U_6$, где $\{P_i\}$ – в виде отдельных векторов матрицы-перестановки, $\{L_i\}$ – нижнее треугольные матрицы и $\{U_i\}$ – верхние унитреугольные матрицы, которые хранятся на месте нижнего поддиагонального треугольника матрицы $A$ и верхнего треугольника $A$ с её поддиагональю;
	\item реализовать метод Гаусса с оптимальным заполнением для разреженных матриц и построением соответствующего этому методу разложения матрицы системы $A$ вида $A=PLUP'$; при реализации учесть формат хранения A; реализовать для этого метода итерационное уточнение простой и переменной точности, оценить качество получаемых результатов;
	\item реализовать метод Гаусса с выбором по столбцу и построить с его помощью оценщик числа обусловленности матрицы системы в строчной норме; протестировать качество данного оценщика и получаемой с его помощью оценкой относительной погрешности решения через вектор невязки.
\end{enumerate}